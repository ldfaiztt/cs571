\documentclass[12pt,letterpaper]{article} % For LaTeX2e
\usepackage{amsmath,amsthm,amsfonts,amssymb,amscd}
\usepackage{fullpage}
\usepackage{graphicx}
\usepackage{lastpage}
\usepackage{listings}
\lstset{
	numbers=left,
	numbersep=5pt,
	stepnumber=1,
	tabsize=2,
	showstringspaces=false
}
\usepackage{enumerate}
\usepackage{fancyhdr}
\usepackage{final_project}
\usepackage{hyperref}
\usepackage{mathrsfs}
\usepackage{cancel}
\usepackage{times}
\usepackage{xcolor}
\usepackage[margin=3cm]{geometry}



\title{Homework 8}


\author{
Matt Dickenson\\
Department of Political Science\\
Duke University\\
Durham, NC 27708 \\
\texttt{mcd31@duke.edu}
}

\newcommand{\fix}{\marginpar{FIX}}
\newcommand{\new}{\marginpar{NEW}}

\nipsfinalcopy

\begin{document}


\maketitle

\begin{abstract}
ABSTRACT HERE
\end{abstract}



\section{Introduction}

\section{Related Work}

\section{Methods}

% Summarize your data, describing the problem you are attempting to solve (e.g., prediction, classification). 

% Describe the features that you will use from your data to solve this problem. 

% Present your mathematical model in a rigorous and clear way with equations when necessary. 

% Please specify the random variables and parameters in your model (for example, xi ∈ R are real-valued and yj are categorical variables), and 

% provide interpretations for these variables (i.e., which are observed, which are hidden, and which you will be estimating, and how each of the observed values will be processed from the original data). 

% In three sentences, describe the inference problem and what methods you will use for this task.

% This work should be approximately 1-2 pages in TeX, and this text can be used directly in your project report. Please feel free to use the final project report TeX template to write this homework up.

\section{Results}

\section{Discussion}

\section{Conclusion}











\subsubsection*{Acknowledgments}

Thanks to Michael D. Ward and the National Science Foundation for supporting this project.

\subsubsection*{References}

% CITE A LOT. Any unreferenced methods, prior work, or biological phenomenon, unless it is textbook-common, will be penalized.

\small{
[1] Alexander, J.A. \& Mozer, M.C. (1995) Template-based algorithms
for connectionist rule extraction. In G. Tesauro, D. S. Touretzky
and T.K. Leen (eds.), {\it Advances in Neural Information Processing
Systems 7}, pp. 609-616. Cambridge, MA: MIT Press.

[2] Bower, J.M. \& Beeman, D. (1995) {\it The Book of GENESIS: Exploring
Realistic Neural Models with the GEneral NEural SImulation System.}
New York: TELOS/Springer-Verlag.

[3] Hasselmo, M.E., Schnell, E. \& Barkai, E. (1995) Dynamics of learning
and recall at excitatory recurrent synapses and cholinergic modulation
in rat hippocampal region CA3. {\it Journal of Neuroscience}
{\bf 15}(7):5249-5262.
}

\end{document}
