\documentclass[12pt,letterpaper]{article}
\usepackage{amsmath,amsthm,amsfonts,amssymb,amscd}
\usepackage{fullpage}
\usepackage{graphicx}
\usepackage{lastpage}
\usepackage{listings}
\lstset{
	numbers=left,
	numbersep=5pt,
	stepnumber=1,
	tabsize=2,
	showstringspaces=false
}
\usepackage{enumerate}
\usepackage{fancyhdr}
\usepackage{hyperref}
\usepackage{mathrsfs}
\usepackage{cancel}
\usepackage{xcolor}
\usepackage[margin=3cm]{geometry}
\setlength{\parindent}{0.0in}
\setlength{\parskip}{0.05in}

% Edit these as appropriate
\newcommand\course{STA561/CS571}
\newcommand\semester{Fall 2013}     % <-- current semester
\newcommand\hwnum{3}                  % <-- homework number
\newcommand\yourname{Matt Dickenson} % <-- your name
\newcommand\login{mcd31}           % <-- your NetID
\newcommand\hwdate{Due: 7 October, 2013}           % <-- HW due date

\newenvironment{answer}[1]{
  \subsubsection*{Problem #1}
}


\pagestyle{fancyplain}
\headheight 35pt
\lhead{\yourname\ \texttt{\login}\\\course\ --- \semester}
\chead{\textbf{\Large Homework \hwnum}}
\rhead{\hwdate}
\headsep 10pt

\begin{document}

\noindent \emph{Homework Notes:} I did not work with anyone else on this homework or refer to resources other than the course notes, textbook, and course Piazza page.

\begin{answer}{1}

\paragraph{A} 
\begin{eqnarray*}
\hat{\mu}_k &=& \sum_{i=1}^n p(z_i=k | \hat{\pi}, \hat{\Sigma}_k)*x_i \\
\hat{\pi}_k &=& \sum_{i=1}^n p(x_i | \hat{\mu}_k \hat{\Sigma}_k)
\end{eqnarray*}

\paragraph{B} These estimates differ slightly from those in the Murphy textbook. In that text, $\hat{\pi}_k=1/K$ for all $k$, whereas here we use information in the data to estimate $\hat{\pi}$. Similarly, in the Murphy book $\mu_k = {1 \over N_k} \sum_{i:z_i=k} x_i$ relies on the indicator funciton, while in the version above $\hat{\mu}$ is a weighted average of the $x_i$'s based on the probability that observation $i$ is in category $k$. Thus, the version above relies more fully on information in the data rather than the \texttt{argmin} of the deviance function.
\end{answer}



\begin{answer}{2} 

\paragraph{A} Let $n_X = \sum_{i=1}^n X_i$. The data likelihood can be written
\begin{eqnarray*}
P(\mathcal{D}|\theta) &=& \prod_{i=1}^n P(X_i | \theta) \\
&=& \prod_{i=1}^n \mu_{Z_t} \times X_t + (1-\mu_{Z_t}) \times (1-X_t) \\
&=& \mu_{Z_t}^{n_X} \times (1-\mu_{Z_t})^{n-n_X}. 
\end{eqnarray*}

\paragraph{B} The complete log likelihood can be written
\begin{eqnarray*}
\ell_c(\theta) &=& \log P(X, Z | \theta) \\
&=& \log ( \prod_{i=1}^n P(X_i, Z_i | \theta)  ) \\
&=& \sum_{i=1}^n (\mu_{Z_t} \times X_t + (1-\mu_{Z_t}) \times (1-X_t)) \times (\pi_{Z_{t-}1} \times Z_t + (1-\pi_{Z_t}) \times (1-Z_{t-1})) \\
&=& (\mu_{Z_t} \times \pi_{Z_{t-}1}  \times {n_X}) + ((1-\mu_{Z_t}) \times (1-\pi_{Z_t}) \times (n-n_X) )
\end{eqnarray*}

\paragraph{C} Omitted.

\paragraph{D} Omitted.

\paragraph{E} Omitted.

\end{answer}

\begin{answer}{3}

\paragraph{A} 
I would expect to have a single cluster ($K=1$) because in a single dimension the mean value should minimize the squared distance from $X$. 

\paragraph{B}  

\paragraph{C}

\end{answer}

\begin{answer}{4}
Omitted.
\end{answer}

% \answer{4} 

\begin{answer}{5}
\paragraph{A} The width of the tree is $\log_2 K$. 

\paragraph{B} Omitted.
\end{answer}

\begin{answer}{6}
Omitted.
\end{answer}

\end{document}
