\documentclass[twoside]{article}

\usepackage{epsfig}

\setlength{\oddsidemargin}{0.25 in}
\setlength{\evensidemargin}{-0.25 in}
\setlength{\topmargin}{-0.6 in}
\setlength{\textwidth}{6.5 in}
\setlength{\textheight}{8.5 in}
\setlength{\headsep}{0.75 in}
\setlength{\parindent}{0 in}
\setlength{\parskip}{0.1 in}

\newcommand{\lecture}[4]{
   \pagestyle{myheadings}
   \thispagestyle{plain}
   \newpage
   \setcounter{page}{1}
   \noindent
   \begin{center}
   \framebox{
      \vbox{\vspace{2mm}
    \hbox to 6.28in { {\bf STA561:~Probabilistic machine learning \hfill} }
       \vspace{6mm}
       \hbox to 6.28in { {\Large \hfill #1 (#2)  \hfill} }
       \vspace{6mm}
       \hbox to 6.28in { {\it Lecturer: #3 \hfill Scribes: #4} }
      \vspace{2mm}}
   }
   \end{center}
   \markboth{#1}{#1}
   \vspace*{4mm}
}

\begin{document}

\lecture{Lecture name}{1/1/13}{Barbara Engelhardt}{Scribe names}

\section{Instructions}

Send your finished notes to the course TAs
({\tt sta561-ta@duke.edu}), including the Latex file and all .eps
or .pdf files that you've created for figures. Double-check that your file compiles with {\tt pdflatex lec1.tex} before you send it.

These notes should be complete, correct, clear, and free of typos.

If you would like an example of a good set of scribe notes in LaTeX, email me.

Thank you!
Barbara

\end{document}

